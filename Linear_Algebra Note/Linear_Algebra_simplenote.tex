\documentclass[11pt, b5paper, oneside]{book}
\usepackage{CTEX}
\usepackage{amsmath, amsthm, amssymb, bm, graphicx, hyperref, mathrsfs}
\usepackage{geometry}
\geometry{b5paper,scale=0.8}
\usepackage{graphicx} %插入图片的宏包
\usepackage{float} %设置图片浮动位置的宏包
\usepackage{subfigure} %插入多图时用子图显示的宏包
\usepackage{amstext} %公式中包含文字的宏包

% 封面部分
\title{{\Huge{\textbf{Linear Algebra }}}\\——Simple Note}
\author{Wu Yutian}
\date{2021.10.9}
\linespread{1.5}
\newtheorem{theorem}{定理}[section]
\newtheorem{definition}[theorem]{定义}
\newtheorem{lemma}[theorem]{引理}
\newtheorem{corollary}[theorem]{推论}
\newtheorem{example}[theorem]{例}
\newtheorem{proposition}[theorem]{命题}
\begin{document}

% 输出封面
\maketitle

% 前言部分
\pagenumbering{roman}
\setcounter{page}{1}

\begin{center}
    \Huge\textbf{前言}
\end{center}~\

在继微积分的简明笔记之后,同时也开始了线性代数的重新学习,并为后面的矩阵论的学习打好基础。

在线性代数的学习过程中主要参考了《Linear Algebra Done Right》这本书,另外同时也参考了它的中文版《线性代数应该这么学》,这本书不同于国内的教材,从向量空间的角度阐述了整个线性代数的知识体系,被国外很多学校作为教材,是一本很经典的书。另外还有一本和它名字很像的书,叫做《Linear Algebra Done Wrong》同样也是被广泛推荐的一本书,后续应该也会看一下,再对这份笔记进行补充。但是由于我手上有《Linear Algebra Done Right》的中文版,因此就从这本书开始看起,笔记的体系结构也是以此书为依据的。

~\\
% flushright可以右对齐
\begin{flushright}
    \begin{tabular}{c}
        Wu Yutian\\
        2021.10.9
    \end{tabular}
\end{flushright}

% 目录部分
\newpage
\pagenumbering{Roman}
\setcounter{page}{1}
\tableofcontents
\newpage
\setcounter{page}{1}
\pagenumbering{arabic}

% 正文开始
% -------------------------chapter 1-------------------------
\chapter{Vector Space}

\section{$R^n \text{与} C^n$}

组(List)的概念:长度为n的组是n个有顺序的元素,用逗号隔开并且两端用括号包起来。(组有时也被称为是向量)

组与集合的不同:组中的元素是有顺序的并且允许重复,而集合中没有顺序,并且元素不能重复。

定义0:0表示长度为n且所有坐标都是0的组,即:\[0=(0,0,...,0)\]

补充 证明存在且唯一的方法:先证明存在一个这样的元素$a_1$(存在性),然后再假设存在另一个元素$a_2$推导$a_2=a_1$(唯一性)。

\section{向量空间的定义}

在向量空间中定义加法和标量乘法:向量空间V在加法和标量乘法下封闭。

\begin{definition}
    向量空间的定义:向量空间就是带有加法和标量乘法的集合V,满足如下性质(拆分开一共8条):

    \emph{交换性(commutativity)}
    \begin{center}
        对所有$u,v \in V$,都有$u+v=v+u$;
    \end{center}
    
    \emph{结合性(associativity)}
    \begin{center}
        对所有$u,v,w \in V$都有$(u+v)+w=u+(v+w)$;

        对所有$v \in V$和$a,b \in F$都有$(ab)v=a(bv)$;
    \end{center}

    \emph{加法单位元(additive identity)}
    \begin{center}
        存在元素$0 \in V$使得对所有的$v \in V$都有$v+0=v$;
    \end{center}

    \emph{加法逆元(additive inverse)}
    \begin{center}
        对每个$v \in V$都存在$w \in V$使得$v+w=0$;
    \end{center}

    \emph{乘法单位元(numltiplicative indentity)}
    \begin{center}
        对所有$v \in V$都有$1v=v$;
    \end{center}

    \emph{分配性质(distributive properties)}
    \begin{center}
        对所有$a \in F$和$u,v \in V$都有$a(u+v)=au+av$;
        
        对所有$a,b \in F$和$v \in V$都有$(a+b)v=av+bv$;
    \end{center}

    注意:因为向量之间没有乘法,所以没有乘法逆元的性质。
\end{definition}

\begin{definition}
    实向量空间和复向量空间:

    \textbf{V}是\textbf{R}上的向量空间,则\textbf{V}为实向量空间;
    
    \textbf{V}是\textbf{C}上的向量空间,则\textbf{V}为复向量空间。
\end{definition}

两个命题:
\begin{itemize}
    \item 1、\textbf{R}是\textbf{R}上的向量空间。(真)
    \item 2、\textbf{R}是\textbf{C}上的向量空间。(假)
\end{itemize}
    
对于这两个命题的说法,前面的\textbf{R}表示的是在其中取向量,后面的\textbf{R}和\textbf{C}表示的是在其中取标量。这样的话,对于命题2,就意味着:需要有对于$\forall r \in \textbf{R}\text{,}\forall z \in \textbf{C}$,都有$rz \in \textbf{R}$,这显然不成立,也就是说命题2不符合封闭性!

空集不是向量空间,空集中没有加法单位元,因此最小的向量空间是:{0}。
    
\section{(补充)数域的定义}

\begin{definition}
    数域\textbf{F}是一个集合,拥有两种运算:加法、乘法。
    
    加法:对于$\forall x,y\in \textbf{F}$,$\exists$唯一的$x+y\in \textbf{F}$与之对应。

    乘法:对于$\forall x,y\in \textbf{F}$,$\exists$唯一的$xy\in \textbf{F}$与之对应。
\end{definition}

对于定义中加法的要求:(类似于向量空间中的要求)
\begin{itemize}
    \item 1、$x+y=y+x$(交换律)
    \item 2、$(x+y)+z=x+(y+z)$(结合律)
    \item 3、$\exists 0 \in \textbf{F}$使得$x+0=x$对$\forall x \in \textbf{F}$成立。
    \item 4、对于$\forall x \in \textbf{F}$, $\exists(-x) \in \textbf{F}$使得$x+(-x) = 0$。
\end{itemize}

对于定义中乘法的要求:
\begin{itemize}
    \item 1、$xy=yx$(交换律)(向量空间中没有乘法交换律)
    \item 2、$(xy)z=x(yz)$(结合律)
    \item 3、$\exists 1 \in \textbf{F}$,$1 \neq 0$,使得$1x=x$对$\forall x \in \textbf{F}$成立。(乘法单位元)
    \item 4、对于$x \in \textbf{F}$, $x \neq 0$,$\exists(\frac{1}{x}) \in \textbf{F}$使得$x(\frac{1}{x}) = 1$。(乘法逆元)(向量空间中没有乘法逆元)
    \item 
\end{itemize}

乘法分配律:
\begin{itemize}
    \item 只有一条:$x(y+z) = xy+xz$
\end{itemize}

\section{子空间}

\begin{definition}
    如果\textbf{V}的子集\textbf{U}(采用与\textbf{V}相同的加法和标量乘法)也是向量空间,则称\textbf{U}是\textbf{V}的子空间。
\end{definition}













\end{document}